\documentclass
  [hyperref={colorlinks = true,linkcolor = blue, 
             citecolor = blue, urlcolor = blue}
  ]{beamer}


%\setbeamercolor{structure}{fg=black,bg=white}

%\setbeamercolor*{palette primary}{fg=black,bg=white}
%\setbeamercolor*{palette secondary}{fg=black,bg=white}
%\setbeamercolor*{palette tertiary}{fg=black,bg=white}
%\setbeamercolor*{palette quaternary}{fg=black,bg=white}

%\setbeamercolor{section in toc}{fg=black,bg=white}
%\setbeamercolor{alerted text}{fg=black, bg=white}

%\setbeamercolor{titlelike}{parent=palette primary}
%\setbeamercolor{frametitle}{bg=black,fg=white}

%\setbeamercolor*{titlelike}{parent=palette primary}

\setbeameroption{show notes on second screen=right}

\usepackage{caption}
\captionsetup[figure]{labelformat=empty}

\setbeamertemplate{navigation symbols}{}
\setbeamertemplate{footline}[frame number]

 
\usepackage[utf8]{inputenc}

\usepackage[newfloat]{minted}

\usepackage{pgf}
\usepackage{tikz}
\usepackage{upquote}
\usepackage{natbib}
\usepackage[export]{adjustbox}
\usepackage{amsmath}
\usepackage{amssymb}
\usepackage{trfrac}

\usetikzlibrary{arrows,automata,fit, shapes.geometric}

\bibliographystyle{abbrvnat}

\newenvironment{code}{\captionsetup{type=listing}}{}
\SetupFloatingEnvironment{listing}{name=Listing}

\newcommand{\Type}{\; \text{Type}}
\newcommand{\ofType}{\; \colon}

\newcommand{\Def}[2]{
  \setbeamercolor{block title}{fg=white,bg=white!25!black}
  \setbeamercolor{block body}{fg=black,bg=white!75!black}
  \begin{block}{#1}#2\end{block}
}

\newcommand{\Ex}[2]{
  \setbeamercolor{block title}{fg=black,bg=white!75!black}
  \setbeamercolor{block body}{fg=white,bg=white!25!black}
  \begin{block}{#1}#2\end{block}
}

\newcommand{\Code}[4]{
  \setbeamercolor{block title}{fg=white,bg=gray!75!black}
  \setbeamercolor{block body}{fg=black,bg=white!80!gray}
  \begin{block}{#2}
               \inputminted[firstline=#4,linenos,firstnumber=1]
               {#1}{#3}\end{block}
}

\newcommand{\Note}[2]{
  \setbeamercolor{block title}{fg=white,bg=yellow!25!black}
  \setbeamercolor{block body}{fg=black,bg=yellow!80!gray}
  \begin{block}{#1}
  #2
  \end{block}
}

\newcommand{\CodePath}[1]{../Code/src/#1}
\newcommand{\SlidePath}[1]{./Slides/#1}
\newcommand{\ImagePath}[1]{./Images/#1}


\setbeamertemplate{items}[square]

\setbeamercolor{structure}{fg=black, bg=white}
\setbeamercolor{frametitle}{fg=structure.bg, bg=structure.fg}

\title{Introduction to Programming Language Theory and the Simply Typed Lambda Calculus \\
       \small slides: 
       \href{https://github.com/donovancrichton/Talks}
            {https://github.com/donovancrichton/Talks}}
\author{Donovan Crichton} 
  
\date{April 2025}

\begin{document}
 
\frame{\titlepage}

\begin{frame}[fragile]
  \frametitle{The goal of this talk}
    \include{\SlidePath{goaloftalk.tex}}
\end{frame}

\begin{frame}[fragile]
  \frametitle{Reading Syntax: Grammar}
    \include{\SlidePath{syntax-grammar.tex}}
\end{frame}

\begin{frame}[fragile]
  \frametitle{Why does Grammar look like this?}
    \include{\SlidePath{why-does-grammar-look-like-this.tex}}
\end{frame}

\begin{frame}[fragile]
  \frametitle{What about in programming?}
    \include{\SlidePath{grammar-in-haskell.tex}}
\end{frame}

\begin{frame}[fragile]
  \frametitle{Church Encoding - Naturals and Booleans}
    \include{\SlidePath{church-encoding.tex}}
\end{frame}

\begin{frame}[fragile]
  \frametitle{Church Encoding - Functions and Predicates}
    \include{\SlidePath{church-functions-and-predicates.tex}}
\end{frame}

\begin{frame}[fragile]
  \frametitle{Reading Syntax: Typing Rules}
    \include{\SlidePath{typing-rules.tex}}
\end{frame}

\begin{frame}[fragile]
  \frametitle{The Program Context}
    \include{\SlidePath{context.tex}}
\end{frame}

\begin{frame}[fragile]
  \frametitle{Example Contexts}
    \include{\SlidePath{example-contexts.tex}}
\end{frame}

\begin{frame}[fragile]
  \frametitle{The Function Type Formation Rule}
    \include{\SlidePath{arrow-formation.tex}}
\end{frame}

\begin{frame}[fragile]
  \frametitle{The Function Term Formation Rule}
    \include{\SlidePath{func-formation.tex}}
\end{frame}

\begin{frame}[fragile]
  \frametitle{The Function Term Elimination Rule}
    \include{\SlidePath{app-elimination.tex}}
\end{frame}

\begin{frame}[fragile]
  \frametitle{STLC: The Simply Typed Lambda Calculus}
\end{frame}

\begin{frame}[fragile]
  \frametitle{STLC 2}
\end{frame}

\begin{frame}[fragile]
  \frametitle{$\alpha$-Equivalence/Conversion.}
\end{frame}

\begin{frame}[fragile]
  \frametitle{Substitution and Computation}
\end{frame}

\begin{frame}[fragile]
  \frametitle{Alpha Equivalence, Free Variables, Beta Reduction}
\end{frame}

\begin{frame}[fragile]
  \frametitle{De-Bruijn Indices}
\end{frame}

\begin{frame}[fragile]
  \frametitle{Lift-and-shift}
\end{frame}

\begin{frame}[fragile]
  \frametitle{A lot of work at the meta-level}
\end{frame}

\begin{frame}[fragile]
  \frametitle{Explicit Substitutions (Paper)}
  \include{\SlidePath{1991explicitsub.tex}}
\end{frame}

\begin{frame}[fragile]
  \frametitle{Extending the STLC with Explicit Substitutions}
\end{frame}

\begin{frame}[fragile]
  \frametitle{Syntax cont.}
\end{frame}

\begin{frame}[fragile]
  \frametitle{ES rules.}
\end{frame}

\begin{frame}[fragile]
  \frametitle{Evaluating ES}
\end{frame}

\begin{frame}[fragile]
  \frametitle{Evaluating ES 2}
\end{frame}

% Main body of talk goes here

% What do I want to talk about?

% 1. Give motivation for talk

% Why do we have explicit substitutions in the first place?
%   - it is more difficult to reason precisely about meta-level operations
%   - disconnect between LC mathematical description and implementation
%   


\begin{frame}[fragile]
\frametitle{References}
\bibliography{references}{}
\end{frame}




\end{document}


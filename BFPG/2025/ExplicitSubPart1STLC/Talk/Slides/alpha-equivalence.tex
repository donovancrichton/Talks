
\only<1>{
\Def{$\lambda(x : \beta).x = \lambda(y : \beta).y$}
    {Two lambda terms are said to be $\alpha$-\textit{equivalent}
     if we can rename bound variables and still have the same term.
     We may $\alpha$-convert or $\alpha$-rename a lambda term into
     another $\alpha$-equivalence term.
    }
}

\only<2>{
\Def{$\lambda(x : \beta).x = \lambda(y : \beta).y$}
    {Two lambda terms are said to be $\alpha$-\textit{equivalent}
     if we can rename bound variables and still have the same term.
     We may $\alpha$-convert or $\alpha$-rename a lambda term into
     another $\alpha$-equivalence term.
    }
\Ex{Examples of $\alpha$-equivalence.}
   {\vspace{-0.5cm}
    \begin{align*}
      \lambda(x : \beta).x &= \lambda(y : \beta).y
    \end{align*}
   }
}

\only<3>{
\Def{$\lambda(x : \beta).x = \lambda(y : \beta).y$}
    {Two lambda terms are said to be $\alpha$-\textit{equivalent}
     if we can rename bound variables and still have the same term.
     We may $\alpha$-convert or $\alpha$-rename a lambda term into
     another $\alpha$-equivalence term.
    }
\Ex{Examples of $\alpha$-equivalence.}
   {\vspace{-0.5cm}
    \begin{align*}
      \lambda(x : \beta).x &= \lambda(y : \beta).y \\
      \lambda(x : \beta).\lambda(x : \beta).x &=
      \lambda(y : \beta).\lambda(x : \beta).x 
    \end{align*}
   }
}

\only<4>{
\Def{$\lambda(x : \beta).x = \lambda(y : \beta).y$}
    {Two lambda terms are said to be $\alpha$-\textit{equivalent}
     if we can rename bound variables and still have the same term.
     We may $\alpha$-convert or $\alpha$-rename a lambda term into
     another $\alpha$-equivalence term.
    }
\Ex{Examples of $\alpha$-equivalence, and non-equivalence.}
   {\vspace{-0.5cm}
    \begin{align*}
      \lambda(x : \beta).x &= \lambda(y : \beta).y \\
      \lambda(x : \beta).\lambda(x : \beta).x &=
        \lambda(y : \beta).\lambda(x : \beta).x \\
      \lambda(x : \beta).\lambda(x : \beta).x &\neq
        \lambda(y : \beta).\lambda(x : \beta).y 
    \end{align*}
   }
}

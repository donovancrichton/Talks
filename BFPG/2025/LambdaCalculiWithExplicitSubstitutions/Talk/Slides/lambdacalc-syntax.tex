\noindent
\begin{minipage}[t]{0.48 \linewidth}
  \onslide<2->{\Def{Our set of variables}
{
\centering
  \onslide<3->{$V ::= x, y, z, \; ...$}
}
}

 \onslide<4->{\Def{Untyped Lambda Calculus Grammar}
{
\vspace{-0.5cm}
\begin{flalign*}
 \onslide<5->{M,N &::= &V\; & \;\;
  \text{\textit{Variable.}} &\\}
 \onslide<6->{&\;\;\;\;\;| &M\;N\; & \;\;
  \text{\textit{Application.}} &\\}
 \onslide<7->{&\;\;\;\;\;| &\lambda V.M\; & \;\;
  \text{\textit{Abstration.}} &}
\end{flalign*}
}
}
\vspace{-1cm}
\end{minipage}
$\;\;$
\noindent
\begin{minipage}[t]{0.47 \linewidth}
  \onslide<8->{\Code{idris}{Code Tie-In}{\CodePath{Lambda.idr}}{3}}
\end{minipage}

\note[item]{A quick refresher on the untyped lambda calculus}
\note[item]{Smallest turing-complete language.}
\note[item]{First we need a set of variables.}
\note[item]{Grammar/Syntax has 3 terms.}
\note[item]{Looks scary? You can read this already, clearly inspires data declarations in ML languages}



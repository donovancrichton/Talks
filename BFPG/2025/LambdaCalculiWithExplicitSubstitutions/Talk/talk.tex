\documentclass
  [hyperref={colorlinks = true,linkcolor = blue, 
             citecolor = blue, urlcolor = blue}
  ]{beamer}

\setbeameroption{show notes on second screen=right}

\usepackage{caption}
\captionsetup[figure]{labelformat=empty}

\setbeamertemplate{navigation symbols}{}
\setbeamertemplate{footline}[frame number]

 
\usepackage[utf8]{inputenc}

\usepackage[newfloat]{minted}

\usepackage{pgf}
\usepackage{tikz}
\usepackage{upquote}
\usepackage{natbib}
\usepackage[export]{adjustbox}
\usepackage{amsmath}
\usepackage{amssymb}
\usepackage{trfrac}

\usetikzlibrary{arrows,automata,fit, shapes.geometric}

\bibliographystyle{abbrvnat}

\newenvironment{code}{\captionsetup{type=listing}}{}
\SetupFloatingEnvironment{listing}{name=Listing}

\newcommand{\Def}[2]{
  \setbeamercolor{block title}{fg=white,bg=orange!75!black}
  \setbeamercolor{block body}{fg=white,bg=orange!25!black}
  \begin{block}{#1}#2\end{block}
}

\newcommand{\Ex}[2]{
  \setbeamercolor{block title}{fg=white,bg=teal!75!black}
  \setbeamercolor{block body}{fg=white,bg=teal!25!black}
  \begin{block}{#1}#2\end{block}
}

\newcommand{\Code}[4]{
  \setbeamercolor{block title}{fg=white,bg=purple!75!black}
  \setbeamercolor{block body}{fg=black,bg=white!80!gray}
  \begin{block}{#2}\inputminted[firstline=#4]{#1}{#3}\end{block}
}

\newcommand{\Note}[2]{
  \setbeamercolor{block title}{fg=white,bg=yellow!25!black}
  \setbeamercolor{block body}{fg=black,bg=yellow!80!gray}
  \begin{block}{#1}
  #2
  \end{block}
}

\newcommand{\CodePath}[1]{../Code/src/#1}
\newcommand{\SlidePath}[1]{./Slides/#1}
\newcommand{\ImagePath}[1]{./Images/#1}


\setbeamertemplate{items}[square]

\title{Lambda Calculi With Explicit Substitutions}
\author{Donovan Crichton}
  
\date{Februrary 2025}

\begin{document}
 
\frame{\titlepage}

\begin{frame}[fragile]
  \frametitle{Preliminaries}
  \begin{itemize}
  \item Slides and Examples available at:
    \href{https://github.com/donovancrichton/Talks}
         {https://github.com/donovancrichton/Talks}
  \item This talk: BFPG/LambdaCalculiWithExplicitSubstituions
  \end{itemize}
  \note{Welcome to the talk!}
\end{frame}

\begin{frame}[fragile]
\frametitle{About me}
  \include{\SlidePath{aboutme.tex}}
\end{frame}


\begin{frame}[fragile]
  \Def{This is a test def$\Sigma$.}
      {A test definition for some concept.}
  \Ex{This is a test example.}
     {An example for some concept.}
  \Code{idris}{The Identity Function}
       {\CodePath{Id.idr}}{4}
\end{frame}

\begin{frame}[fragile]
  \frametitle{Untyped Lambda Calculus Syntax}
  \include{\SlidePath{lambdacalc-syntax.tex}}
\end{frame}

\begin{frame}[fragile]
  \frametitle{Untyped Lambda Calculus Computation}
\end{frame}

\begin{frame}[fragile]
  \frametitle{The Problem}
  \include{\SlidePath{problemstatement.tex}}
\end{frame}


\begin{frame}[fragile]
  \frametitle{Explicit Subsitutions (Paper)}
  \include{\SlidePath{1991explicitsub.tex}}
\end{frame}

% Main body of talk goes here

% What do I want to talk about?

% 1. Give motivation for talk

% Why do we have explicit substitutions in the first place?
%   - it is more difficult to reason precisely about meta-level operations
%   - disconnect between LC mathematical description and implementation
%   


\begin{frame}[fragile]
\frametitle{References}
\bibliography{references}{}
\end{frame}




\end{document}


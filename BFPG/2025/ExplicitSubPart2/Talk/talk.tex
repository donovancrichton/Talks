\documentclass
  [hyperref={colorlinks = true,linkcolor = blue, 
             citecolor = blue, urlcolor = blue}
  ]{beamer}


%\setbeamercolor{structure}{fg=black,bg=white}

%\setbeamercolor*{palette primary}{fg=black,bg=white}
%\setbeamercolor*{palette secondary}{fg=black,bg=white}
%\setbeamercolor*{palette tertiary}{fg=black,bg=white}
%\setbeamercolor*{palette quaternary}{fg=black,bg=white}

%\setbeamercolor{section in toc}{fg=black,bg=white}
%\setbeamercolor{alerted text}{fg=black, bg=white}

%\setbeamercolor{titlelike}{parent=palette primary}
%\setbeamercolor{frametitle}{bg=black,fg=white}

%\setbeamercolor*{titlelike}{parent=palette primary}

%\setbeameroption{show notes on second screen=right}

\usepackage{caption}
\captionsetup[figure]{labelformat=empty}

\setbeamertemplate{navigation symbols}{}
\setbeamertemplate{footline}[frame number]

 
\usepackage[utf8]{inputenc}

\usepackage[newfloat]{minted}

\usepackage{pgf}
\usepackage{tikz}
\usepackage{upquote}
\usepackage{natbib}
\usepackage[export]{adjustbox}
\usepackage{amsmath}
\usepackage{amssymb}
\usepackage{trfrac}
\usepackage[normalem]{ulem}

\usetikzlibrary{arrows,automata,fit, shapes.geometric, quotes
 ,positioning}

\bibliographystyle{abbrvnat}

\newenvironment{code}{\captionsetup{type=listing}}{}
\SetupFloatingEnvironment{listing}{name=Listing}

\newcommand{\Type}{\; \text{Type}}
\newcommand{\ofType}{\; \colon}

\newcommand{\Def}[2]{
  \setbeamercolor{block title}{fg=white,bg=white!25!black}
  \setbeamercolor{block body}{fg=black,bg=white!75!black}
  \begin{block}{#1}#2\end{block}
}

\newcommand{\Ex}[2]{
  \setbeamercolor{block title}{fg=black,bg=white!75!black}
  \setbeamercolor{block body}{fg=white,bg=white!25!black}
  \begin{block}{#1}#2\end{block}
}

\newcommand{\Code}[4]{
  \setbeamercolor{block title}{fg=white,bg=gray!75!black}
  \setbeamercolor{block body}{fg=black,bg=white!80!gray}
  \begin{block}{#2}
               \inputminted[firstline=#4,linenos,firstnumber=1]
               {#1}{#3}\end{block}
}

\newcommand{\Note}[2]{
  \setbeamercolor{block title}{fg=white,bg=yellow!25!black}
  \setbeamercolor{block body}{fg=black,bg=yellow!80!gray}
  \begin{block}{#1}
  #2
  \end{block}
}

\newcommand{\CodePath}[1]{../Code/src/#1}
\newcommand{\SlidePath}[1]{./Slides/#1}
\newcommand{\ImagePath}[1]{./Images/#1}


\setbeamertemplate{items}[square]

\setbeamercolor{structure}{fg=black, bg=white}
\setbeamercolor{frametitle}{fg=structure.bg, bg=structure.fg}

\title{Explicit Substitutions Part 2: \\ Explicit Substitutions \\
       \small slides: 
       \href{https://github.com/donovancrichton/Talks}
            {https://github.com/donovancrichton/Talks}}
\author{Donovan Crichton} 
  
\date{August 2025}

\begin{document}
 
\frame{\titlepage}

\begin{frame}[fragile]
  \frametitle{The goal of this talk}
    \include{\SlidePath{goals.tex}}
\end{frame}

\begin{frame}[fragile]
  \frametitle{Untyped LC}
    \include{\SlidePath{utlc.tex}}
\end{frame}

\begin{frame}[fragile]
  \frametitle{STLC Grammar}
    \include{\SlidePath{stlc-grammar.tex}}
\end{frame}

\begin{frame}[fragile]
  \frametitle{STLC Rules}
    \include{\SlidePath{stlc-typing-rules.tex}}
\end{frame}

\begin{frame}[fragile]
  \frametitle{Explicit Substitution Grammar}
    \include{\SlidePath{explicit-subs-grammar.tex}}
\end{frame}

\begin{frame}[fragile]
  \frametitle{Explicit Subs: Terms}
    \include{\SlidePath{explicit-subs-terms.tex}}
\end{frame}

\begin{frame}[fragile]
  \frametitle{Explicit Subs: Subs}
    \include{\SlidePath{explicit-subs-subs.tex}}
\end{frame}

\begin{frame}[fragile]
  \frametitle{Example: Identity}
    \include{\SlidePath{explicit-subs-id.tex}}
\end{frame}

\begin{frame}[fragile]
  \frametitle{Example: Weakening (Shift)}
    \include{\SlidePath{explicit-subs-weakening.tex}}
\end{frame}

\begin{frame}[fragile]
  \frametitle{Example: Extension (Snoc)}
    \include{\SlidePath{explicit-subs-extension.tex}}
\end{frame}

\begin{frame}[fragile]
  \frametitle{Example: Composition}
    \include{\SlidePath{explicit-subs-composition.tex}}
\end{frame}

\begin{frame}[fragile]
  \frametitle{A Note On Notation: Composition and Context}
    \include{\SlidePath{explicit-subs-notation-composition.tex}}
\end{frame}

\begin{frame}[fragile]
  \frametitle{Explicit Subs: Context Typing}
    \include{\SlidePath{explicit-subs-context-typing.tex}}
\end{frame}

\begin{frame}[fragile]
  \frametitle{Explicit Subs: Type Typing}
    \include{\SlidePath{explicit-subs-type-typing.tex}}
\end{frame}

\begin{frame}[fragile]
  \frametitle{Explicit Subs: Term Typing}
    \include{\SlidePath{explicit-subs-term-typing.tex}}
\end{frame}

\begin{frame}[fragile]
  \frametitle{Explicit Subs: Substitution Typing}
    \include{\SlidePath{explicit-subs-substitution-typing.tex}}
\end{frame}

\begin{frame}[fragile]
  \frametitle{Equational Theory: Category Laws}
    \include{\SlidePath{explicit-subs-category-laws.tex}}
\end{frame}

\begin{frame}[fragile]
  \frametitle{Equational Theory: Closure Laws}
    \include{\SlidePath{explicit-subs-closure-laws.tex}}
\end{frame}

\begin{frame}[fragile]
  \frametitle{Equational Theory: Weakening Laws}
    \include{\SlidePath{explicit-subs-weakening-laws.tex}}
\end{frame}

\begin{frame}[fragile]
  \frametitle{Equational Theory: Distribution Laws}
    \include{\SlidePath{explicit-subs-distribution-laws.tex}}
\end{frame}

\begin{frame}[fragile]
  \frametitle{Equational Theory: Extension Laws}
    \include{\SlidePath{explicit-subs-extension-laws.tex}}
\end{frame}

\begin{frame}[fragile]
  \frametitle{Equational Theory: Computation Laws}
    \include{\SlidePath{explicit-subs-computation-laws.tex}}
\end{frame}

\begin{frame}[fragile]
  \frametitle{Equational Theory Intuition}
    \include{\SlidePath{equational-theory-intuition.tex}}
\end{frame}

\begin{frame}[fragile]
  \frametitle{Conclusions}
    \include{\SlidePath{conclusion.tex}}
\end{frame}

\end{document}


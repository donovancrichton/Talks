\only<1>{
  \textcolor{black}{If the derivations (true statements) hold
                    above the line (premise), 
                    \\ then the derivations hold below the 
                    line (conclusion).}
  \Def{The Type Formation Rule}
      {
    \[ \boxed{\trfrac[(\textit{Ty-Arrow})]
             {\qquad \qquad \qquad \qquad}
             {\qquad \qquad \qquad \qquad}}
    \]
      }
}

\only<2>{
  \textcolor{black}{...from an arbitrary context, $\Gamma$
                   }
  \Def{The Type Formation Rule}
      {
    \[ \trfrac[(\textit{Ty-Arrow})]
          {\boxed{\Gamma} \qquad \qquad \qquad}
          {\qquad \qquad \qquad \qquad}
    \]
      }
}

\only<3>{
  \textcolor{black}{...we may derive (produce, obtain...)
                   }
  \Def{The Type Formation Rule}
      {
    \[ \trfrac[(\textit{Ty-Arrow})]
          {\Gamma \boxed{\vdash} \qquad \qquad \qquad}
          {\qquad \qquad \qquad \qquad}
    \]
      }
}

\only<4>{
  \textcolor{black}{...an arbitrary type, called $A$}
  \Def{The Type Formation Rule}
      {
    \[ \trfrac[(\textit{Ty-Arrow})]
          {\Gamma \vdash \boxed{A \Type} \qquad \qquad \qquad}
          {\qquad \qquad \qquad \qquad}
    \]
      }
}

\only<5>{
  \textcolor{black}{AND}
  \Def{The Type Formation Rule}
      {
    \[ \trfrac[(\textit{Ty-Arrow})]
          {\Gamma \vdash A \Type \boxed{\qquad} \qquad \qquad}
          {\qquad \qquad \qquad \qquad}
    \]
      }
}

\only<6>{
  \textcolor{black}{...From the same arbitrary context $\Gamma$, extended with a variable 'x' of type $A$}
  \Def{The Type Formation Rule}
      {
    \[ \trfrac[(\textit{Ty-Arrow})]
          {\Gamma \vdash A \Type \qquad \boxed{\Gamma, x \ofType A} \qquad \qquad}
          {\qquad \qquad \qquad \qquad}
    \]
      }
}

\only<7>{
  \textcolor{black}{...we may derive}
  \Def{The Type Formation Rule}
      {
    \[ \trfrac[(\textit{Ty-Arrow})]
          {\Gamma \vdash A \Type 
           \qquad \Gamma, x \ofType A \boxed{\vdash} 
           }
          {\qquad \qquad \qquad \qquad}
    \]
      }
}

\only<8>{
  \textcolor{black}{some arbitrary type $B$.}
  \Def{The Type Formation Rule}
      {
    \[ \trfrac[(\textit{Ty-Arrow})]
          {\Gamma \vdash A \Type 
           \qquad \Gamma, x \ofType A \vdash \boxed{B \Type}
           }
          {\qquad \qquad \qquad \qquad}
    \]
      }
}

\only<9>{
  \textcolor{black}{Then, from that arbitrary $\Gamma$ (program scope)}
  \Def{The Type Formation Rule}
      {
    \[ \trfrac[(\textit{Ty-Arrow})]
          {\Gamma \vdash A \Type 
           \qquad \Gamma, x \ofType A \vdash B \Type
           }
          {\boxed{\Gamma} \qquad \qquad \qquad}
    \]
      }
}

\only<10>{
  \textcolor{black}{...we may derive}
  \Def{The Type Formation Rule}
      {
    \[ \trfrac[(\textit{Ty-Arrow})]
          {\Gamma \vdash A \Type 
           \qquad \Gamma, x \ofType A \vdash B \Type
           }
          {\Gamma \boxed{\vdash} \qquad \qquad \qquad}
    \]
      }
}

\only<11>{
  \textcolor{black}{...an arrow type between them}
  \Def{The Type Formation Rule}
      {
    \[ \trfrac[(\textit{Ty-Arrow})]
          {\Gamma \vdash A \Type 
           \qquad \Gamma, x \ofType A \vdash B \Type
           }
          {\Gamma \vdash \boxed{A \rightarrow B} \qquad \qquad}
    \]
      }
}

\only<12>{
  \textcolor{black}{...that is also a type.}
  \Def{The Type Formation Rule}
      {
    \[ \trfrac[(\textit{Ty-Arrow})]
          {\Gamma \vdash A \Type 
           \qquad \Gamma, x \ofType A \vdash B \Type
           }
          {\Gamma \vdash A \rightarrow B \boxed{\Type}}
    \]
      }
}

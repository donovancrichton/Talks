\only<1>{
\Def{A Note On Notation: Composition and Context}
    {
     \textbf{Two notations for composition exist in the literature:}
     \begin{itemize}
       \item \textbf{Sequential (Abadi et al. original):} $\gamma \circ \delta$ means "first $\gamma$, then $\delta$"
       \item \textbf{Categorical (most papers):} $\delta \circ \gamma$ means "first $\gamma$, then $\delta$"
       \item In this talk: We use $\gamma \circ \delta$ meaning "first $\delta$, then $\gamma$"
     \end{itemize}
    }
\Ex{Why the difference?}
    {The original paper used sequential notation matching evaluation order.\\
     Modern presentations use categorical notation matching function composition.}
}

\only<2>{
\Def{A Note On Notation: Composition and Context}
    {
     \textbf{Critical: Pay attention to context order in substitution typing!}
     \vspace{0.3cm}
     
     In the judgment $\Gamma \vdash \gamma : \Delta$:
     \begin{itemize}
       \item $\Delta$ is the \textbf{source} context (domain)
       \item $\Gamma$ is the \textbf{target} context (codomain)  
       \item $\gamma$ maps FROM $\Delta$ TO $\Gamma$
     \end{itemize}
    }
\Ex{Composition typing rule}
    {$\trfrac{\Gamma \vdash \gamma : \Delta \qquad \Delta \vdash \delta : \Theta}
            {\Gamma \vdash \gamma \circ \delta : \Theta}$
     
     The middle context $\Delta$ must match!\\
     Read: $\gamma \circ \delta$ takes us from $\Theta$ to $\Gamma$ via $\Delta$}
}

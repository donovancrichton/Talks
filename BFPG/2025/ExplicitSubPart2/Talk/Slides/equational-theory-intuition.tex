\only<1>{
\Def{Equational Theory Intuition}
    {
     \begin{itemize}
       \item Equations let us rewrite expressions to canonical forms
       \item Substitutions can be simplified before application
       \item No need for complex variable renaming ($\alpha$-conversion)
     \end{itemize}
    }
}

%\only<2>{
%\Def{Equational Theory Intuition}
%    {
%     \begin{itemize}
%       \item Equations let us rewrite expressions to canonical forms
%       \item Substitutions can be simplified before application
%       \item No need for complex variable renaming (α-conversion)
%       \item Evaluation becomes simpler: just apply equations
%     \end{itemize}
%    }
%\Ex{Example Simplification}
%    {$M[id] = M$ \quad (identity does nothing)\\
%     $M[\gamma][\delta] = M[\gamma \circ \delta]$ \quad (compose first)}
%}

%\only<3>{
%\Def{Why Explicit Substitutions Matter}
%    {
%     By making substitution explicit in the calculus:
%     \begin{itemize}
%      \item Evaluation logic becomes deterministic
%      \item Proofs about substitution are internalized
%       \item Implementation matches theory directly
%       \item Variable capture is impossible by construction
%     \end{itemize}
%    }
%}

%\only<4>{
%\Def{Key Insight}
%    {
%     Traditional approach: Substitution is a meta-operation\\
%     \vspace{0.3cm}
%     Explicit approach: Substitution is part of the syntax\\
%     \vspace{0.3cm}
%     \centering
%     \fbox{Substitutions are first-class citizens with their own type system!}
%    }
%}
